\documentclass[9pt]{beamer}
%
\input{preamble}
%
\title{Exploiting the Laws of Order in Smart Contracts}
\subtitle{\small An appetizer of the work by~\citet{10.1145/3293882.3330560}}
\date{\today}
\author{Marco Ciccalè Baztán (\href{mailto:m.ciccale@alumnos.upm.es}{m.ciccale@alumnos.upm.es})}
\institute{Analysis of Concurrent and Distributed Systems,\\%
  MSc in Formal Methods in Computer Science,\\%
  Universidad Politécnica de Madrid}
\titlegraphic{\hfill\includegraphics[height=2.2cm]{include/upm-T.png}}

\begin{document}
%
\maketitle
%
\begin{frame}{Original Work}
  \centering
  These slides are based on the original work by~\citet{10.1145/3293882.3330560}:
  %
  %
  \medskip
  %

  \textsf{\bfseries Exploiting the Laws of Order in Smart Contracts}
\end{frame}
%
\bgroup
\let\oldfootnoterule\footnoterule
\def\footnoterule{\only<5->\oldfootnoterule}
\begin{frame}{Motivation}
  \begin{itemize}
  \item \alert{Smart contracts} (or \alert{contracts}) are stateful
    programs (code $+$ data) that are \emph{stored} in the
    \emph{blockchain} in a replicated fashion.
  %
  \pause\item \alert{Transactions} (invocations of \emph{contract code}) are
    totally ordered, as agreed by the majority of miners.
  %
  \pause\item Thus, contracts can be used as \emph{automatic} and
    \emph{trustworthy} \alert{mediators}.
    %
    E.g., \emph{voting}, \emph{distribution of rewards}, \ldots
  %
  \pause\item Contracts can be simultaneously invoked by multiple users at
    once, and their execution order is \alert{non-deterministic}.
  %
    \pause\item Since contracts deal with money (in the form of
    \emph{cryptocurrencies}), attacks exploiting blockchains' or
    contracts' vulnerabilities often result in \alert{hundreds of
      millions of dollars'} being \emph{stolen} or
    \emph{lost}.\footnote<5->{See, for example,~\citep{dao}.}
  %
\pause\item Furthermore, contracts \emph{cannot} be \alert{patched} after
    deployment, which gives rise to their audit and analysis \emph{before}
    their deployment.
  \end{itemize}
\end{frame}
\egroup
%
\begin{frame}{Problem Statement}
  \begin{itemize}
  \item This work focuses on \emph{detecting} a class of
    vulnerabilities related to the contract's inherent
    \alert{concurrent} execution model.
  %
  \pause\item I.e., detecting if the re-ordering of events (function
    invocations) can lead to different outcomes, which they call
    \alert{event ordering} (EO) bugs.
  %
  \pause\item The challenge lies on how to explore the
    \alert{combinatorial} search-space of event orderings
    for finding EO bugs.
  %
  \pause\item For a typical contract with $20$ $2$-ary functions,
  the number of traces composed of $6$ events is $\approx 2^{38}$.
  \emoji{fearful-face}
  \end{itemize}
  %
  \pause\vspace{1cm}
  \begin{center}
    \alert{DPOR}\ldots? \emoji{thinking-face} We'll see later!
  \end{center}
\end{frame}
%
\begin{frame}[fragile]{Illustrative Example: Contract \texttt{Casino} with an asynchronous callback}
\begin{lstlisting}[mathescape=true,language=Solidity,basicstyle=\footnotesize\ttfamily]
contract Casino { // 100 Ethers!
  function bet() payable {
    // make sure we can pay out the player
    if (address(this).balance < msg.value * 100) throw; $\label{code1:betcheck}$
    // asynchronous invokation for random number
    bytes32 oid = oraclize_query(...);$\label{code1:calloracle}$
    bets[oid] = msg.value; $\label{code1:storebet1}$
    players[oid] = msg.sender; $\label{code1:storebet2}$ } $\label{code1:endbet}$
  function __callback(bytes32 myid, string result) $\label{code1:callback}$
    onlyOraclize onlyIfNotProcessed(myid) {
    ...
    if (parseInt(result) % 200 == 0) $\label{code1:wincheck}$
    players[myid].send(bets[myid] * 100); $\label{code1:payout}$ }...}
\end{lstlisting}
  %
  \smallskip
  %
  Consider players \texttt{1} and \texttt{2}, both betting $1$ Ether:
  %
  \pause\smallskip
  %
  \begin{center}\small
  \begin{tabular}{cc}
      \textbf{Trace 1}
    & \textbf{Trace 2}\\\hline\hline
    %
      \texttt{bet}$_{\texttt{1}}$\texttt{()}
    & \texttt{bet}$_{\texttt{1}}$\texttt{()}\\
    %
      \texttt{bet}$_{\texttt{2}}$\texttt{()}
    & \texttt{\_\_callback}\texttt{(1,0)}\\
    %
      \texttt{\_\_callback}\texttt{(1,0)}
    & \texttt{bet}$_{\texttt{2}}$\texttt{()}\\
    %
      \texttt{\_\_callback}\texttt{(2,0)}
    & \texttt{\_\_callback}\texttt{(2,0)}\\\hline\hline
    %
    \pause
    %
      Yields insufficient balance
    & Declines the bet\\[-.5mm]
    %
      after paying off player $1$
    & placed by player $2$\\[-.5mm]
    %
      (Line~\ref{code1:payout})
    & (Line~\ref{code1:betcheck})
  \end{tabular}
  \end{center}
  %
\end{frame}
%
% \begin{frame}[fragile]{Illustrative Examples (II): Contract \texttt{IOU} with on-chain transaction ordering}
%   \begin{minipage}{.7\textwidth}
% \begin{lstlisting}[mathescape=true,language=Solidity,basicstyle=\scriptsize\ttfamily]
% contract IOU {
%  // Approves the transfer of tokens
%  function approve(address _spender, uint256 _val) { $\label{code2:approve}$
%    allowed[msg.sender][_spender] = _val;
%    return true;  }
%  // Transfers tokens
%  function transferFrom(address _from, address _to,$\label{code2:transferFrom}$
%                        uint256 _val) {
%    require(allowed[_from][msg.sender] >= _val
%            && balances[_from] >= _val
%            && _val > 0);
%    balances[_from] -= _val;
%    balances[_to] += _val;
%    allowed [_from][msg.sender] -= _val; $\label{code2:updatebalance}$
%    return true;  }...}
% \end{lstlisting}
%   \end{minipage}
%   %
%   \begin{minipage}{.25\textwidth}
    
%   \end{minipage}

% \end{frame}
\begin{frame}{Overview of the Approach: How to reduce the search space?}
  \begin{itemize}
  \item Dynamic testing approach for finding concrete
    witnesses.\footnote{These witnesses would then be executed by
      \emph{human} analysts for confirming or denying the existence of
      an EO bug.}
  %
  \pause\item Given that functions are usually written in a certain
    way s.t.~they \emph{induce} an \textbf{execution order relation},
    capture this order between functions as a \alert{happens-before}
    ($\mathit{hb}$) relation.
    %
    \pause
    %
    This way, if we find that two events $e_1$ and $e_2$ are in
    $\mathit{hb}(e_1,e_2)$, it is possible to prune all permutations
    that violate such ordering.  We'll see later how these
    $\mathit{hb}$ relations can be induced.
  %
  \pause\item A trivial but effective insight is that events that
    don't read / write global state can be \alert{reordered
      arbitrarily}, thus they don't need to be taken into account when
    finding $\mathit{hb}$ relations.
  %
  \pause\item When possible, \alert{linearize} traces with
    asynchronous callbacks and treat them as ``canonical'' traces.
  \end{itemize}
\end{frame}
%
\begin{frame}{Contract Concurrency Model}
  \begin{itemize}
  \item Events are executed \alert{atomically} and
    \alert{deterministically}.
  %
  \pause\item A miner runs a function---conceptually as a
    \emph{single threaded} transaction, with its given inputs until it
    terminates \alert{succesfully} or by an \alert{exception}.
  %
  \pause\item The source of \emph{concurrency} lies in the nondeterministic
    choices of each miner while ordering the transactions of a block.  
  \end{itemize}
  \[\rho \estep{e_1} \rho'\]
  \[\rho \tstep{e_1} \rho'\]
\end{frame}
%
\begin{frame}{Formalization (I): Contracts, messages and events}
  \begin{definition}[Contract]
    A contract $c$ is a tuple $\langle \mathsf{id}, M, \rho \rangle$
    where $\mathsf{id}$ is the contract identifier, $M$ its source
    code, and $\rho$ its state, i.e., the balance and fields.
  \end{definition}
  %
  \pause
  %
  \begin{definition}[Message]
    A message $m$ is a tuple
    $\langle \mathsf{sender}, \mathsf{to}, \mathsf{value},
    \mathsf{fname}, \mathsf{data} \rangle$ representing the sender,
    receiver, the amount of Ethers, the function name to invoke, and
    its paremeters, respectively.
  \end{definition}
  %
  \pause
  %
  \begin{definition}[Event]
    An event $e$ for a contract $c$ is a message $m$ whose
    $\mathsf{fname}$ field is a function name from the code of $c$.
  \end{definition}
\end{frame}
%
\begin{frame}{Formalization (II): Traces and event-ordering bugs}
  \begin{definition}[Trace]
    We use $\rho \estep{e} \rho'$ to indicate the run of an event
    $e$ from a state $\rho$ to a state $\rho'$.
    %
    Naturally, $\rho_1 \tstep{h} \rho_n$ denotes the sequence of event
    runs $\rho_1 \estep{e_1} \ldots \estep{e_n} \rho_n$
    s.t.~$h = [e_1,\ldots,e_n]$.
    %
    And, $\rho_1 \tstep{h}$ \Lightning\ denotes a sequence of event runs
    that ends by raising an exception.
  \end{definition}
  %
  \pause
  %
  \begin{definition}[Event-Ordering Bug]
    Given a contract $c$, a pair of \emph{valid} event traces
    $h = [e_1,\ldots,e_n]$ and $h' = [e_1',\ldots,e_n']$
    constitutes an \emph{event-ordering bug} iff:
    %
    \begin{itemize}
    \item $h$ is a \emph{permutation} of $h'$, i.e., a reordering,
    \item if $\rho_0 \tstep{h} \rho_n$ and
      $\rho_0 \tstep{h'} \rho'_n$, then $\rho_n \neq \rho'_n$.
    \end{itemize}
  \end{definition}
\end{frame}
%
\begin{frame}{\textsc{EthRacer}'s Algorithm}
  \begin{enumerate}
  \item<1-> \textbf{Symbolic event analysis}
    %
    \begin{enumerate}
    \item<4-> \textbf{Syntactic analysis} to recover the \emph{public functions} of the
    contract from the bytecode.  These are the \alert{entrypoints} of the
    contract.
    %
    \item<5-> Using standard \textbf{dynamic symbolic execution}, for each entrypoint
      $e_i$ of the contract, it gathers \alert{constraints} (as SMT formul\ae) down
      all paths that over-approximate the set of values down each of the paths.
    \end{enumerate}
  %
  \item<2-> \textbf{Concretization}
    %
    \begin{enumerate}
    \item<6-> Check the \alert{feasibility} of each path constraints
      using ``off-the-shelf'' SMT solvers for helping \emph{path
        exploration}.
    %
    \item<7-> Generate \alert{concrete} value(s) for each path
      constraints so that they can then be used for \emph{fuzzing} the
      contract.  These values are also generated using SMT solvers,
      and they are lifted to concrete events that can be directly
      executed by the EVM.
    \end{enumerate}
  \item<9-> \alert{\bfseries Happens-before relations!}
  \item<3-> \textbf{Fuzzing}
    %
    \begin{enumerate}
    \item<8-> From the concrete values computed before, the contract
      is \alert{fuzzed} for finding EO bugs, which are then flagged as
      witnesses.
    \end{enumerate}
  \end{enumerate}
\end{frame}
%
\begin{frame}{\textit{Happens-Before} Relation}
\end{frame}
%
\begin{frame}{Linearizability in Callbacks}
\end{frame}
%
\begin{frame}{Experimental Evaluation}
\end{frame}
%
\begin{frame}{Example Results}
\end{frame}
%
\begin{frame}{Comparison to Related Work}
\end{frame}
%
\begin{frame}{Limitations and Future Work}
\end{frame}
%
\begin{frame}{Conclusions}
\end{frame}
%
\begin{frame}{}
  \centering\huge\sffamily
  \textbf{\structure{Thank You!\\Questions?}}
\end{frame}
%
\begin{frame}[allowframebreaks]{}
  \scriptsize
  \bibliography{biblio.bib}
  \bibliographystyle{abbrvnat}
\end{frame}
%
\end{document}

%%% Local Variables:
%%% mode: LaTeX
%%% TeX-master: t
%%% TeX-engine: luatex
%%% End:
