\documentclass[9pt]{beamer}
%
\input{preamble}
%
\title{Exploiting the Laws of Order in Smart Contracts}
\subtitle{\small An appetizer of the work by~\citet{10.1145/3293882.3330560}}
\date{\today}
\author{Marco Ciccalè Baztán (\href{mailto:m.ciccale@alumnos.upm.es}{m.ciccale@alumnos.upm.es})}
\institute{Analysis of Concurrent and Distributed Systems,\\%
  MSc in Formal Methods in Computer Science,\\%
  Universidad Politécnica de Madrid}
\titlegraphic{\hfill\includegraphics[height=2.2cm]{include/upm-T.png}}

\begin{document}
%
\maketitle
%
\begin{frame}{Original Work}
  \centering
  These slides are based on the original work by~\citet{10.1145/3293882.3330560}.
\end{frame}
%
\bgroup
\let\oldfootnoterule\footnoterule
\def\footnoterule{\only<4->\oldfootnoterule}
\begin{frame}{Motivation}
  \begin{itemize}
  \item \alert{Smart contracts} (or \alert{contracts}) are stateful
    programs (code $+$ data) that are \emph{stored} in the
    \emph{blockchain} in a replicated fashion.
  %
  \pause\item \alert{Transactions} (invocations of \emph{contract code}) are
    totally ordered, as agreed by the majority of miners.
  %
  \pause\item Thus, contracts can be used as \emph{automatic} and
    \emph{trustworthy} \alert{mediators}.
    %
    E.g., \emph{voting}, \emph{distribution of rewards}, \ldots
  %
  \pause\item Contracts can be simultaneously invoked by multiple users at
    once, and their execution order is \alert{non-deterministic}.
  %
  \pause\item Since contracts deal with money (in the form of
    \emph{cryptocurrencies}), attacks exploiting vulnerabilities of
    the blockchain or contracts often result in \alert{hundreds of
      millions of dollars'} being \emph{stolen} or
    \emph{lost}.\footnote<4->{See, for example,~\citep{dao}.}
  %
\pause\item Furthermore, contracts \emph{cannot} be \alert{patched} after
    deployment, which gives rise to their audit and analysis \emph{before}
    their deployment.
  \end{itemize}
\end{frame}
\egroup
%
\begin{frame}{Problem Statement}
  \begin{itemize}
  \item This work focuses on \emph{detecting} a class of
    vulnerabilities related to the contract's inherent
    \alert{concurrent} execution model.
    %
    More specifically, if the re-ordering of transactions can lead to
    different outcomes (reminds of partial order right?), which they call
    \alert{event ordering} (EO) bugs.
  %
  \pause\item The challenge is the \alert{combinatorial blowup} of
    the search-space for finding EO bugs.
  \end{itemize}
  %
  \pause\vspace{1cm}
  \begin{center}
    \alert{DPOR}\ldots? \emoji{thinking-face} We'll see later!
  \end{center}
\end{frame}
%
\begin{frame}[fragile]{Illustrative Examples (I): Contract \texttt{Casino} with an asynchronous callback}
\begin{lstlisting}[mathescape=true,language=Solidity,basicstyle=\scriptsize\ttfamily]
contract Casino {
  function bet() payable {
    // make sure we can pay out the player
    if (address(this).balance < msg.value * 100) throw; $\label{code1:betcheck}$
    bytes32 oid = oraclize_query(...); // asynchronous invokation for random number$\label{code1:calloracle}$
    bets[oid] = msg.value; $\label{code1:storebet1}$
    players[oid] = msg.sender; $\label{code1:storebet2}$ } $\label{code1:endbet}$
  function __callback(bytes32 myid, string result) $\label{code1:callback}$
    onlyOraclize onlyIfNotProcessed(myid) {
    ...
    if (parseInt(result) % 200 == 42) $\label{code1:wincheck}$
    players[myid].send(bets[myid] * 100); $\label{code1:payout}$ }...}
\end{lstlisting}
  %
  \bigskip
  %
  \begin{itemize}
  \pause\item Consider players 1 and 2, both betting 1 Ether.
  \pause\item The trace $\langle\texttt{bet}_1; \texttt{bet}_2; \texttt{\_\_callback}_1; \texttt{\_\_callback}_2\rangle$
    allows both bets to be placed, but if both win, player 2 will not be paid anything; whereas,
  \pause\item the trace $\langle\texttt{bet}_1; \texttt{\_\_callback}_1; \texttt{bet}_2; \texttt{\_\_callback}_2\rangle$
    would not let player 2 place the bet.
  \end{itemize}
\end{frame}
%
% \begin{frame}[fragile]{Illustrative Examples (II): Contract \texttt{IOU} with on-chain transaction ordering}
%   \begin{minipage}{.7\textwidth}
% \begin{lstlisting}[mathescape=true,language=Solidity,basicstyle=\scriptsize\ttfamily]
% contract IOU {
%  // Approves the transfer of tokens
%  function approve(address _spender, uint256 _val) { $\label{code2:approve}$
%    allowed[msg.sender][_spender] = _val;
%    return true;  }
%  // Transfers tokens
%  function transferFrom(address _from, address _to,$\label{code2:transferFrom}$
%                        uint256 _val) {
%    require(allowed[_from][msg.sender] >= _val
%            && balances[_from] >= _val
%            && _val > 0);
%    balances[_from] -= _val;
%    balances[_to] += _val;
%    allowed [_from][msg.sender] -= _val; $\label{code2:updatebalance}$
%    return true;  }...}
% \end{lstlisting}
%   \end{minipage}
%   %
%   \begin{minipage}{.25\textwidth}
    
%   \end{minipage}

% \end{frame}
%
\begin{frame}{Key Insight}
\end{frame}
%
\begin{frame}{Main Contributions}
\end{frame}
%
\begin{frame}{\textsc{EthRacer} Overview}
\end{frame}
%
\begin{frame}{\textsc{EthRacer} Analysis Approach}
\end{frame}
%
\begin{frame}{\textit{Happens-Before} Relation}
\end{frame}
%
\begin{frame}{Linearizability in Callbacks}
\end{frame}
%
\begin{frame}{Experimental Evaluation}
\end{frame}
%
\begin{frame}{Example Results}
\end{frame}
%
\begin{frame}{Comparison to Related Work}
\end{frame}
%
\begin{frame}{Limitations and Future Work}
\end{frame}
%
\begin{frame}{Conclusions}
\end{frame}
%
\begin{frame}{}
  \centering\huge\sffamily
  \textbf{\structure{Thank You!\\Questions?}}
\end{frame}
%
\begin{frame}[allowframebreaks]{}
  \scriptsize
  \bibliography{biblio.bib}
  \bibliographystyle{abbrvnat}
\end{frame}
%
\end{document}

%%% Local Variables:
%%% mode: LaTeX
%%% TeX-master: t
%%% End:
